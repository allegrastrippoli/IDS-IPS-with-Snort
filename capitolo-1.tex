\section{Sicurezza e vulnerabilità della rete}

Quando si realizza un sistema di difesa si procede per livelli, si utilizzano strumenti di difesa passivi e poi, a seconda della necessità, si fa affidamento a tecnologie più evolute (ma più costose) per il controllo del traffico e per la prevenzione di attacchi. Come si possono strutturare meccanismi di difesa intelligenti che riconoscano i pattern di attacco e adottino delle misure per contrastarli? Il primo passo è acquisire una conoscenza profonda delle possibili minacce e vulnerabilità della rete. In questo capitolo non viene offerta una panoramica completa, bensì uno scorcio che permetta di comprendere quali siano nel concreto le motivazioni che spingono gli utenti a introdurre sistemi di difesa.

\subsection{Minacce comuni in rete}

Per minacce "comuni" si intendono tutte quelle a cui un utente potrebbe andare incontro con un  utilizzo sistematico dei principali servizi in rete. Un utente naviga siti web, gestisce la propria posta elettronica, utilizza pennette usb, esegue il download di file etc... e non sempre è consapevole di esporsi a dei rischi.
Gli attacchi più comuni da cui difendersi sono ~\cite{softwarelab} :

\vspace{1cm}

Virus, worm e altri agenti automatici, che si diffondono autonomamente tramite meccanismi di infezione. Alla fase di propagazione segue quella di attivazione, causata da un comportamento umano, da un processo schedulato oppure potrebbe avvenire una auto-attivazione. Una volta che un malware ha attaccato un sistema vulnerabile può mostrare i suoi effetti immediatamente oppure può stare nascosto anche per molto tempo prima di causare i danni. Esistono diversi tipi di malware come Spyware (raccoglie dati e monitora attività dell'utente) Ransomware (effettua una crittografia dei dati e li tiene in "ostaggio") Trojan (che come un cavallo di Troia si presenta come un normale programma ma in realtà è stato realizzato per uno scopo malevolo). Il firewall e software antivirus possono essere due strumenti di difesa.

\vspace{1cm}

Scan su larga scala alla ricerca indiscriminata di sistemi vulnerabili, sfruttando buchi di sicurezza generalmente noti. Gli scan potrebbero rilevare vulnerabilità soprattutto nei servizi non aggiornati recentemente. Una vulnerabiltà potrebbe essere sfruttata per tentare un rce (remote code execution) che permette all'autore dell'attacco di eseguire codice da remoto sul computer vittima e tramite un escalation dei privilegi acquisire il controllo della macchina (ossia accesso come root).

\vspace{1cm}

Il phishing consiste nel cercare di acquisire con l’inganno le informazioni personali degli utenti, sfruttando email e siti web con loghi falsificati. Il termine phishing ricorda il termine "fishing", proprio perché come nella pesca vengono utilizzate delle esche per irretire le proprie vittime. Il phishing consente di accedere alle credenziali di accesso, ai dati carte di credito etc... È un tipo di attacco molto popolare e diffuso, fa leva sull'incapacità di molti utenti di non saperlo riconoscere (spesso sembrano email urgenti e quindi si è incentivati ad aprirle, oppure gli url sono molto simili a quelli di siti ufficiali e quindi non è facile capire che sia falso).

\vspace{1cm}

Un exploit è un frammento di codice o software che sebbene non sia di per sé dannoso, può contenere un payload maligno. Gli exploit sono generalmente ospitati su siti web compromessi, un utente non attento potrebbe finire su un sito non sicuro, oppure potrebbe esservi reindirizzato aprendo un'email sospetta (esempio di phishing). Spesso vengono creati dei patch per contrastare gli exploit e correggere le vulnerabilità, quindi è sufficiente aggiornare i servzi installati ed il sistema operativo per limitare questo tipo di attacchi. Un esempio di payload maligno è una backdoor. Le backdoor "porte di servizio" rappresentano degli accessi privilegiati al sistema in grado di superare le procedure di sicurezza attivate. In questo modo autenti non autorizzati sono in grado di accedere a un pc all’insaputa dell’utente.

\vspace{1cm}

Un attacco DDoS (Distributed Denial of Service) consiste nell'inviare un flusso continuo di traffico falso ai servizi online, come i siti web, fino a quando non si sovraccaricano e crollano. Si distingue dall'attacco DoS (Denial of Service) perché il traffico è proveniente da più fonti e non da un singolo dispositivo. Si può descrivere come un bombardamento di traffico al servizio, il quale non è attrezzato per gestire l'enorme volume di trafficco e collassa. In questo modo gli utenti reali che hanno necessità di accedere ad una risorsa di rete sono impossibilitati. Più aumenta il numero di fonti da cui parte la minaccia e più è difficile identificare l'autore primario. Esistono diverse tipologie di DDoS, può trattarsi di un attacco indiscriminato, mirato a colpire un protocollo o addirittura una specifica vulnerablità di un'applicazione web. Ad accumunarli è il concetto di "flood" (inondazione) e le modalità con cui vengono realizzati, spesso tramite una rete di bot.


\section{Difesa della rete}

Per evitare attacchi, accessi al sistema e connessioni non autorizzate bisogna adottare delle misure che restringano il campo in cui queste intromissioni possano avvenire. Il principale punto di accesso ad un sistema è tramite una porta TCP o UDP aperta, e quindi è necessario individuare quali porte sia effettivamente necessario tenere aperte e trovare un criterio, uno strumento, che permetta di chiudere qualsiasi altro punto di accesso all'host in rete. È funzionale a questo scopo introdurre il concetto di firewall.


\subsection{Firewall}

Il firewall è uno strumento di difesa che implementa una policy di sicurezza che permette di determinare quale traffico può transitare. I Firewall vengono classificati secondo due tipologie:

• Host-based se controlla tutto il traffico in entrata e in uscita generato da un host.

• Newtork-based se filtra tutto il traffico in entrata e in uscita da una sottorete.

Ad esempio se si immaginano due sottoreti, una interna (lan), una esterna (internet), il firewall viene posto in mezzo, in modo da venire attraversato dal traffico che la lan e internet si scambiano. Utilizzando delle regole il firewall filtra il traffico ed esattamente come un muro blocca tutti i pacchetti indesiderati.

\subsection{Firewall con iptables}

Sebbene ci siano diversi modi per realizzare un firewall è oggetto di analisi solo il firewall realizzato con le regole iptables. Come prima cosa bisogna settare la policy, in ACCEPT (se un pacchetto non matcha nessuna regola viene accettato) oppure in DROP (se un pacchetto non matcha nessuna regola viene scartato). Se viene scelta una politica in ACCEPT le regole saranno in DROP, e viceversa. È sconsigliato utilizzare una politica in ACCEPT perchè il firewall potrebbe risultare più "lasco" rispetto ad uno che adotta una politica DROP, tuttavia con una politica DROP bisogna stare attenti a non dimenticarsi di far passare tutto il traffico autorizzato.
Ad esempio se ci si connette in ssh ad un server e si modifica il firewall (che utilizza una politica in DROP) dimenticando la rule (in ACCEPT) relativa all'ssh, la connessione con il server viene repentinamente interrotta.
Dopo aver scelto la policy è necessario scrivere le regole specificando se si tratta regole di input-output, forwarding, prerouting o postrouting, il protocollo (TCP o UDP), l'interfaccia, la porta, lo stato in cui è ammessa la connessione.

• Una regola è di INPUT-OUTPUT se riguarda il traffico in ingresso o in uscita dal firewall.

• Una regola è di FORWARDING se i pacchetti hanno come origine internet/lan, come destinazione la lan/internet e sono solo di passaggio per il firewall, che agisce da router e instrada i pacchetti. I pacchetti viaggiano attraverso catene di forwarding, serve una chain per ogni "flusso" di traffico. Ad esempio serve una catena per i pacchetti che viaggiano dalla lan verso internet, e un'altra dalla rete esterna verso quella interna.

• Una regola di prerouting lavora sui pacchetti in entrata al sistema e applica delle regole ancora prima che i pacchetti vengano matchati con le regole di routing.

• Una regola di postrouting lavora sui pacchetti in uscita dal sistema e applica delle regole dopo che i pacchetti sono stati matchati con le regole di routing. Nelle regole di prerouting va specificato DNAT, mentre nelle regole di postrouting SNAT per permettere il mapping degli indirizzi ip.

Per quanto riguarda invece gli stati di una connessione sono ammessi i pacchetti in uno dei seguenti stati:

• NEW - una nuova richiesta di connessione da parte di un client (SYN TCP o nuovo pacchetto UDP)

• ESTABLISHED - pacchetti relativi a connessioni già stabilite

• RELATED - pacchetti correlati a connessioni esistenti e ESTABILSHED, come ad esempio il traffico ftp

Ecco degli esempi di regole relativi alla porta 443 (che viene usata dal protocollo HTTPS):

\vspace{1cm}

iptables -A INPUT -p tcp --sport 443 -i ens160 -m state --state ESTABLISHED -j ACCEPT

\vspace{1cm}

Questo è un esempio di regola di INPUT permette di accettare i pacchetti in input di tipo ESTABLISHED sull'interfaccia ens160 del firewall, protocollo tcp, source port 443 (ossia permette di accettare risposte da parte di un server https)


\vspace{1cm}

iptables -A ie -p tcp --dport 443 -m state --state NEW,ESTABLISHED -j ACCEPT

\vspace{1cm}

Questo è un esempio di regola di FORWARDING, "ie" è il nome della catena che consente il transito di pacchetti dalla rete interna verso quella esterna. Come nell'esempio delle regole input output è specificata la porta, questa volta si tratta di una destination port. Il significato della regola è il seguente: è permesso il forward dei pacchetti di tipo NEW ed ESTABLISHED dalla lan verso internet.

\vspace{1cm}

iptables -t nat -A PREROUTING -i ens160 -s 0/0 -d 10.222.111.2 -p tcp --dport 443 -j DNAT --to 50.50.50.3:443

\vspace{1cm}

Questo è un esempio di regola di PREROUTING. Il significato della regola è: tutti i pacchetti che arrivano sull'interfaccia ens160 del firewall, che hanno come source internet (0/0) e come destinazione l'ip 10.222.111.2 (ip associato all'ens160) verranno forwardati sull'host che ha ip  50.50.50.3 sulla porta 443.


\vspace{1cm}

Non c'è in quest'analisi la pretesa di mostrare nei particolari il funzionamento delle regole iptables, ma solo di fornirne un'idea generale.
Se si fa rifermineto alla pila ISO-OSI queste permettono un buon grado di dettaglio dei livelli di rete (liv. 3) e di trasporto (liv. 4), ma non di distinguere i livelli applicativi (HTTP, FTP etc...). Il firewall dunque ha dei limiti. Si ambisce alla realizzazione di un apparato più intelligente, che sfrutti la conoscenza dei livelli superiori della pila ISO-OSI per analizzare i pacchetti ed eventualmente intraprendere delle azioni attive a difesa del sistema. Per questo il firewall si accosta spesso a sistemi di rilevazione e prevenzione delle intrusioni, i cosiddetti IDS e IPS.


\subsection{IDS e  IPS}

IDS, Intrusion Detection System é un dispositivo software o hardware che viene utilizzato per identificare attività anomale, accessi non autorizzati a un computer o a una rete di computer. L'intercettazione dei pacchetti sospetti avviene principalmente monitorando il traffico e verificando i log di sistema. Mettere in piedi un meccanismo di controllo di questo tipo aumenta le possibilità di poter riconoscere pattern di attacco noti, possibili scan e alte minacce ... in modo da poter reagire tempestivamente. Gli IDS possono anche sfruttare database, librerie e regole per rilevare le intrusioni. Quando un'attività di rete corrisponde a una regola nota all'ids, questo segnala il tentativo di intrusione. Il limite principale è che l'affidabilità di questo strumento dipende interamente dalla tempestività con cui il database degli attacchi viene aggiornato.
Quando gli IDS rilevano una violazione della sicurezza la notificano generando degli alert, ed è poi compito dell'amministratore (un utente che ha i privilegi di root) fermare l'attività sospetta. Un dispositivo che richiede meno user assistance poichè prende autonomamente delle decisioni per contrastare possibili attacchi è un IPS, un Intrusion Prevention System, che permette di isolare i pacchetti sospetti interrompendo connessioni e bloccando IP. Questo obiettivo può essere raggiunto riprogrammando la lista di controllo degli accessi del firewall in modo dinamico.
Alcuni limiti degli IDS sono l'incapacità di analizzare pacchetti cifrati (non è possibile estrarre il contenuto del pacchetto senza conoscere la chiave) e la necessità (se si basabo su regole) di dover essere costantemente aggiornati per far fronte anche alle nuove minacce.

\subsection{Cosa deve garantire un buon IDS e un buon IPS}

Gli IDS e IPS sfruttano database, librerie e set di regole per decidere se un pacchetto vada bloccato oppure no. Non sono strumenti infallibili e possono commettere degli errori "di valutazione". La qualità di un buon IDS e IPS dipende dal numero di falsi positivi e falsi negativi che rileva. Un falso positivo si verifica quando un IDS/IPS classifica come maligna un'attività che invece dovrebbe essere autorizzata. Ad esempio i pacchetti danneggiati lungo il percorso o generati da software con bug o problemi possono essere riconosciuti erroneamente come tentativi di intrusione e considerati come pacchetti malevoli. Il problema dei falsi positivi è meno dannoso di quello dei falsi negativi, con un dovuto intervento vi si può porre rimedio (ad esempio modificando le regole dell'IDS o del firewall). Al contrario tutta l'attività sospetta che supera controlli, analisi dei pacchetti e raggiunge l'ip destinatario indisturbato può generare una quantità di danni variabile, fino a compromettere l'integrità dell'intero sistema. I falsi negativi potrebbero introdurre virus, provocare malfunzionamenti, perdita di dati. Prevenire i casi di falsi negativi è il principale obiettivo di un sistema di difesa.

\subsection{Tecniche usate}

Così come il firewall implementa una policy di sicurezza, anche gli IDS e gli IPS utilizzano delle strategie, in questo caso strategie di detection. Si dividono in tre tipi:

• Signature-based detection, che si basa su signatures "firme". Una firma è un metodo fi rilevamento che si basa sulla presenza di segni o caratteristiche distintive in un exploit. Sono specificamente progettate per rilevare exploit noti poiché contengono segni distintivi, come stringhe , offset fissi, informazioni di debug, che possono essere correlati o meno allo sfruttamento effettivo di una vulnerabilità. Le signature devono essere preconfigurate e sono statiche, ovvero non cambiano se non vengono aggiornate. Si tratta del metodo più semplice ed efficace per rilevare gli attacchi noti. Questo tipo di rilevamento è in genere classificato come "day after detection", poiché sono necessari veri exploit pubblici affinché questo tipo di rilevamento funzioni. Le aziende antivirus utilizzano questo tipo di tecnologia per proteggere i propri clienti dalle epidemie di virus. Come abbiamo visto nel corso degli anni, questo tipo di protezione ha solo capacità di protezione limitate poiché il virus ha già infettato qualcuno prima che sia possibile scrivere una firma.

• Behaviour-based detection, per cui il sistema è in grado di confrontare il comportamento noto/normale con quello sconosciuto/anormale. Il vantaggio è poter rilevare efficacemente nuovi attacchi e vulnerabilità non viste prima. È meno dipendente dai protocolli e dal sistema operativo, e facilita la rilevazione di abusi di privilegi. Tuttavia bisogna superare la difficoltà di dover istruire il sistema in modo corretto.

• Stateful protocol analysis detection (o policy-based detection). Stateful significa che viene considerato e tracciato lo stato dei protocolli. I pacchetti quindi vengono analizzati nell’insieme, e non solo singolarmente. Questo metodo identifica le anomalie degli stati del protocollo, e considera benigni solo le attività conformi agli standard dei protocolli internazionali.

~\cite{cybersecurity360}.

\subsection{Tipi di IDS}

Un NIDS, Network Intrusion Detection System è un sistema signature-based che monitora il traffico da e verso la lan. I pacchetti che matchano le rules vengono registrati in alert generati dal sistema. Un NIDS configura la scheda di rete per funzionare in modo “promiscuo”, il che significa che saranno fatti passare i pacchetti attraverso lo stack di rete indipendentemente dal fatto che siano o meno indirizzati a una particolare macchina. NIDS verificha il payload dei pacchetti (e a volte anche gli header) per stabilire che non siano presenti tracce di codice maligno. Gli exploit, i virus, i worm e altro software maligno generano traffico di rete che se intercettato e studiato, può smascherare l'attacco prima che questo abbia generato danni. Ad esempio una specifica stringa in un payload potrebbe permettere di identificare un exploit, così come specifiche opzioni negli header del TCP/IP potrebbero rivelare un altro genere di attacco. Se viene rilevata l’attività maligna, qualcuno produrrà una “signature”. Sulla base di queste signature, il motore di ricerca (uno dei componenti principali dell'IDS che si occupa di elaborare e analizzare i pacchetti applicando le regole) dell’IDS deciderà se contrassegnare un pacchetto come potenzialmente maligno. Al contrario di un NIDS, un HIDS, Host-based Intrusion Detection System indaga il traffico di un singolo dispositivo endpoint, e non dell'intera sottorete, per il resto il funzionamento si può considerare analogo al NIDS.

~\cite{zerounoweb}.

\subsection{Tipi di IPS}

Un NIPS, Network Intrusion Prevention System che come i NIDS utilizza una strategia di detection signature-based e si occupa di controllare il traffico da e verso la lan. A differenza del NIDS però intraprende azioni attive a discapito dei pacchetti sospetti. Oltre a segnalare attività anomale all'amministratore, un pacchetto può essere dropped, rejected e un NIPS potrebbe bloccare un IP mittente che non considera affidabile.
Analogamente agli HIDS esistono gli HIPS, Host-based Intrusion Prevention System, utilizzano una strategia di detection signature-based e proteggono attivamente il flusso di traffico da un singolo dispositivo endpoint bloccando il transito di pacchetti sospetti.
Una strategia behaviour-based è invece implementata dall'NBA, Network Behaviour Analysis. A differenza di un NIPS richiede un periodo di formazione (noto anche come "baseline") per apprendere il traffico normale ed essere così in grado di distinguerlo dal traffico anomalo. Identificare flussi di traffico insoliti può essere un ottimo modo per prevenire attacchi DoS e DDoS. Uno dei parametri per confrontare i vari sistemi di prevenzione delle intrusione è in base al numero di falsi postivi e falsi negativi. Un sistema NBA addestrato potrebbe essere più performante di un normale IPS. Tuttavia un NBA mal istruito potrebbe raggiungere un risultato opposto, ossia potrebbe non essere grado di distinguere attività benigne da quelle maligne.
Un sistema invece che sfrutta l'analisi dei protocolli (stateful protocol analysis detection) è il WIPS, Wireless Intrusion Prevention System. Lo scopo principale di un WIPS è impedire l'accesso non autorizzato alle reti locali da parte di dispositivi wireless.



