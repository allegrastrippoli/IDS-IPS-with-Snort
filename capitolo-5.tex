\section{Prerequisiti e installazione}

Come prerequisito è necessario installare delle librerie che permettono di catturare pacchetti.

\begin{verbatim}
    yum install -y gcc flex bison zlib libpcap pcre libdnet tcpdump   
    yum install -y libnghttp2  
\end{verbatim}

Poi viene installato Snort.

\begin{verbatim}
    yum install snort.x86_64
\end{verbatim}

Per aggiornare le librerie installate:

\begin{verbatim}
    ldconfig
\end{verbatim}

Snort su CentOS viene installato nella directory /usr/local/bin/snort, è buona norma creare un link simbolico nel path /usr/sbin/snort.

\begin{verbatim}
    ln -s /usr/local/bin/snort /usr/sbin/snort  
\end{verbatim}

Per eseguire Snort in modo sicuro senza accedere come root, bisogna creare un nuovo utente senza privilegi e un nuovo gruppo di utenti.

\begin{verbatim}
groupadd snort
useradd snort -r -s /sbin/nologin -c SNORT_IDS -g snort
\end{verbatim}

\section{Configurazione}

Per adeguare Snort alle esigenze della rete bisogna:

• modificare il file di configurazione.

• importare le regole.

• gestire i log.

Le principali directory utilizzate da Snort sono:

\texttt{/etc/snort}: in questa directory si trova il file di configurazione snort.conf e la sottodirectory che contiene le rules.

\texttt{/var/log/snort}: in questa directory vengono memorizzati i log, si possono specificare anche altri path, questo è quello di default.

Bisogna assicurarsi che lo user snort e il gruppo snort abbiano i permessi di lettura/ scrittura/ esecuzione in questo modo:

\begin{verbatim}
    chmod -R 5775 /etc/snort
    chmod -R 5775 /var/log/snort
    chmod -R 5775 /usr/local/lib/snort_dynamicrules
    chown -R snort:snort /etc/snort
    chown -R snort:snort /var/log/snort
    chown -R snort:snort /usr/local/lib/snort_dynamicrules
 \end{verbatim}

Viene creato un file per le local.rules: \texttt{vi /etc/snort/rules/local.rules }

Ora si arriva al cuore della configurazione. Bisogna modificare il file snort.conf in modo da soddisfare i requisiti della rete. Per farlo si accede al file con il comando \texttt{vi /etc/snort/snort.conf}
Snort deve sniffare e analizzare i pacchetti sull'interfaccia esterna della vm firewall, per fare questo bisogna definire la variabile HOME-NET e assegnargli il range di IP da proteggere 10.222.111.2/24. Viene creata anche una seconda variabile EXTERNAL-NET per indicare la rete esterna, in questo caso rappresentata da tutte le altre lan e internet.

\begin{verbatim}
ipvar HOME_NET 10.222.111.2/24
ipvar EXTERNAL_NET !$HOME_NET
\end{verbatim}

Vengono create delle variabili che definiscano il path delle rules.

\begin{verbatim}
var RULE_PATH /etc/snort/rules
var SO_RULE_PATH /etc/snort/so_rules
var PREPROC_RULE_PATH /etc/snort/preproc_rules
\end{verbatim}

Qui viene specificato il formato dei log, in questo caso utilizzando unified2 e tcpdump

\begin{verbatim}
# unified2
output unified2: filename merged.log, limit 128, nostamp
# pcap
output log_tcpdump: tcpdump.log
\end{verbatim}

Per includere le local.rules:

\begin{verbatim}
include RULE_PATH/local.rules
\end{verbatim}

Manca un ultimo passaggio: includere le community rules.

\section{Community Rules}

Per scaricare le community rules:

\begin{verbatim}
wget https://www.snort.org/rules/community -O ~/community.tar.gz
\end{verbatim}

Per estrarre le regole e copiarle nella directory dove si trovano le rules:

\begin{verbatim}
tar -xvf ~/community.tar.gz -C ~/
cp ~/community-rules/* /etc/snort/rules
\end{verbatim}

Per commentare tutte le regole diverse da quelle della community:

\begin{verbatim}
sed -i 's/include RULE_PATH/#include RULE_PATH/' /etc/snort/snort.conf
\end{verbatim}

Per includere tutte le regole della community in /etc/snort/snort.conf

\begin{verbatim}
include $RULE_PATH/snort.rules
\end{verbatim}

\section{Pulled Pork}

Ora è possibile procedere con l'installazione di PulledPork il plugin che permette di scaricare automaticamente le nuove regole della community.

\begin{verbatim}
yum -y install pulledpork
\end{verbatim}

Lo script Perl si trova nella directory /usr/local/bin/ e bisogna verfificare che sia eseguibile:

\begin{verbatim}
 chmod +x /usr/local/bin/pulledpork.pl
\end{verbatim}

È possibile verficare il corretto funzionamento eseguendo PulledPork:

\begin{verbatim}
/usr/local/bin/pulledpork.pl –V
\end{verbatim}

Il file di configurazione si trova in \texttt{/etc/snort}, si chiama pulledpork.conf e va modificato:

\texttt{vi /etc/snort/pulledpork.conf}

Viene specificato il path delle community rules da aggiornare:

\begin{verbatim}
rule_path=/usr/local/etc/snort/rules/snort.rules
\end{verbatim}

Pulled pork richiede anche il path delle local.rules per costruire dei file sid-msg.map che contengono informazioni sui metadati (msg) delle local.rules

\begin{verbatim}
local_rules=/usr/local/etc/snort/rules/local.rules
\end{verbatim}

I file sid-msg.map si trovano qui:
\begin{verbatim}
sid_msg=/usr/local/etc/snort/sid-msg.map
\end{verbatim}

viene specificato anche il path del file di configurazione di Snort:
\begin{verbatim}
config_path=/usr/local/etc/snort/snort.conf
\end{verbatim}

Pulled pork aggiorna una blocklist di ip pubblici bloccati dalla comunità:
\begin{verbatim}
block_list=/usr/local/etc/snort/rules/iplists/default.blocklist
\end{verbatim}

\section{Il demone snortd}

Per far si che Snort lavori in background viene attivato il demone snortd. I demoni sono dei programmi eseguiti in background, cioè senza il controllo diretto di un utente. Un sistema di difesa ha l'esigenza di catturare pacchetti e analizzare il traffico durante tutto l'arco di vita della macchina (esposta in rete). Per questo viene largamente sfruttata questa possibilità.

\begin{verbatim}
systemctl daemon-reload
systemctl start snortd
\end{verbatim}

Per avere la conferma che l'avvio del demone sia andata a buon fine:

\begin{verbatim}
systemctl status snortd
\end{verbatim}

\section{FWSnort}

FWsnort è uno script perl che traduce le regole di Snort in iptables. Alcune regole non hanno una traduzione diretta, tuttavia circa il 65 per cento delle rules possono essere tradotte con successo utilizzando l'iptables string match module (modulo che confronta l'inizio della stringa che compone la regola con la chiave di una mappa. Se viene trovata una corrispondenza restituisce l'oggetto corrispondete, un iptables). FWsnort analizza anche la policy iptables in esecuzione sulla macchina per determinare quali regole Snort sono applicabili.

Per installare fwsnort:

\begin{verbatim}
cd /usr/local/src
wget http://www.cipherdyne.org/fwsnort/download/fwsnort-1.0.tar.bz2 
wget http://www.cipherdyne.org/fwsnort/download/fwsnort- 1.0.tar.bz2.md5
wget http://www.cipherdyne.org/fwsnort/download/fwsnort- 1.0.tar.bz2.asc
\end{verbatim}

Per estrarre ed eseguire il file a partire dal pacchetto bz2:

\begin{verbatim}
tar xfj fwsnort-1.0.tar.bz2
cd /usr/local/src/fwsnort-1.0 
./install.pl
\end{verbatim}

Per far partire un'istanza di fwsnort:

\begin{verbatim}
fwsnort
\end{verbatim}

FWSnort installato su un sistema operativo che offre il supporto per il string match module nel kernel, procederà per la traduzione delle regole in iptables, mostrando da cli quante regole sono state convertite correttamente.

Il file di configurazione si trova nella directory \texttt{/etc/fwsnort/fwsnort.conf}

FWSnort gestisce tutto il traffico che attraversa la macchina locale ed è diretto verso/proviene dalla lan. Nel file di configurazione vengono definite delle variabili che corrispondo a ip/porte e vengono utilizzate nelle rules.

\begin{verbatim}
HOME_NET                10.222.111.0/24;
EXTERNAL_NET            !$HOME_NET;

HTTP_SERVERS            $HOME_NET;
DNS_SERVERS             $HOME_NET;

SSH_PORTS               [64022,65022];
HTTP_PORTS              80;
SHELLCODE_PORTS         !80;
\end{verbatim}

Un esempio di community rules:

\begin{verbatim}
alert tcp $EXTERNAL_NET any -> $HTTP_SERVERS $HTTP_PORTS (msg:"SERVER-APACHE Apache Tomcat view source attempt"; flow:to_server,established; content:"%252ejsp"; http_uri; metadata:ruleset community, service http; reference:bugtraq,2527; reference:cve,2001-0590; classtype:web-application-attack; sid:1056; rev:16;)
\end{verbatim}