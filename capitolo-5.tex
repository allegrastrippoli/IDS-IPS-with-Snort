Dopo una prima analisi dei requisiti (capitolo 3) è emerso un problema di sicurezza che viene attenuato in parte dalla presenza di un firewall, ma questo non è sufficiente. Per aggiungere un altro livello di sicurezza si vuole fare affidamento ad un IDS/IPS. Questi sistemi hanno un buon grado di conoscenza dei livelli applicativi, (a differenza del firewall la cui competenza si ferma ai livelli 3 e 4 della pila ISO-OSI) possono investigare i pacchetti e sono istruiti in modo da riconoscere eventuali pattern di attacco noti sfruttando delle strategie di detection (capitolo 2). Si procede quindi con la realizzazione di un sistema IDS/IPS con Snort sulla vm firewall.

\section{Prerequisiti e installazione}

Prima di procedere è necessario installare delle librerie che permettono di catturare i pacchetti. Queste librerie rappresentano un prerequisito e vengono utilizzate per la cattura dei pacchetti che poi verranno analizzati da Snort.

\begin{verbatim}
    yum install -y gcc flex bison zlib libpcap pcre libdnet tcpdump   
    yum install -y libnghttp2  
\end{verbatim}

Dopo questi passaggi preliminari si può passare all'effettiva installazione con yum:

\begin{verbatim}
    yum install snort.x86_64
\end{verbatim}

\section{Configurazione}

Per adeguare Snort alle esigenze della rete bisogna modificare il file di configurzione, importare le regole, gestire i log. Inizialmente è necessario aggiornare le le librerie con il comando:

\begin{verbatim}
    ldconfig
\end{verbatim}

Snort su CentOS viene installato nella directory /usr/local/bin/snort, è buona norma creare un link simbolico a /usr/sbin/snort.

\begin{verbatim}
    ln -s /usr/local/bin/snort /usr/sbin/snort  
\end{verbatim}

Per eseguire Snort su CentOS in modo sicuro senza accesso come root, bisogna creare un nuovo utente senza privilegi e un nuovo gruppo di utenti.

\begin{verbatim}
groupadd snort
useradd snort -r -s /sbin/nologin -c SNORT_IDS -g snort
\end{verbatim}

Le principali directory che sono state create automaticamente usando il comando yum sono:

\texttt{/etc/snort}: in questa directory si trova il file di configurazione snort.conf e la sottodirectory che contiene le rules

\texttt{/var/log/snort}: in questa directory vengono memorizzati i log, si possono specificare anche altri path ma questo è quello di default

Bisogna assicurarsi che lo user snort e il gruppo snort abbiano i permessi di lettura/ scrittura/ esecuzione

\begin{verbatim}
    chmod -R 5775 /etc/snort
    chmod -R 5775 /var/log/snort
    chmod -R 5775 /usr/local/lib/snort_dynamicrules
    chown -R snort:snort /etc/snort
    chown -R snort:snort /var/log/snort
    chown -R snort:snort /usr/local/lib/snort_dynamicrules
 \end{verbatim}

Viene creata uno file per le local.rules: \texttt{vi /etc/snort/rules/local.rules }

Ora si arriva al cuore della configurazione. Bisogna modificare il file snort.conf in modo da soddisfare i requisiti della rete. Per farlo viene eseguito il comando \texttt{vi /etc/snort/snort.conf}
Vengono create due variabili, HOME-NET ed EXTERNAL-NET per distinguere gli indirizzi che appartengono alla lan da tutto il resto (altre lan/ internet). La HOME NET corrisponde all'ip dell'interfaccia esterna della vm firewall, perché è lì che Snort sniffa i pacchetti per analizzarli. L'EXTERNAL-NET invece corrisponde a tutti gli altri ip.

\begin{verbatim}
ipvar HOME_NET 10.222.111.2/24
ipvar EXTERNAL_NET !$HOME_NET
\end{verbatim}

In ogni file di configurazione va specificato il path di dove si trovano le regole.

\begin{verbatim}
var RULE_PATH /etc/snort/rules
var SO_RULE_PATH /etc/snort/so_rules
var PREPROC_RULE_PATH /etc/snort/preproc_rules
\end{verbatim}

Qui viene specificato il formato dei log, in questo caso utilizzando unified2 e tcpdump

\begin{verbatim}
# unified2
output unified2: filename snort.log, limit 128
# pcap
output log_tcpdump: tcpdump.log
\end{verbatim}

Per includere le local.rules:

\begin{verbatim}
include RULE_PATH/local.rules
\end{verbatim}

Manca un ultimo passaggio: includere le community rules. Per farlo è prima necessario scaricarle con una wget.

\section{Community Rules}

Per scaricare le community rules:

\begin{verbatim}
wget https://www.snort.org/rules/community -O ~/community.tar.gz
\end{verbatim}

Per estrarre le regole e copiarle nella cartella di configurazione:

\begin{verbatim}
tar -xvf ~/community.tar.gz -C ~/
cp ~/community-rules/* /etc/snort/rules
\end{verbatim}

Per commentare tutte le regole diverse da quelle della community

\begin{verbatim}
sed -i 's/include RULE_PATH/#include RULE_PATH/' /etc/snort/snort.conf
\end{verbatim}

Per includere tutte le regole della community di nuovo in /etc/snort/snort.conf

\begin{verbatim}
include $RULE_PATH/snort.rules
\end{verbatim}


\section{Pulled Pork}

Ora è possibile procedere con l'installazione di PulledPork il plugin che permette di scaricare automaticamente le nuove regole della community.

\begin{verbatim}
yum -y install pulledpork
\end{verbatim}

Lo script Perl si trova nella directory /usr/local/bin/ e bisogna verfificare che sia eseguibile:

\begin{verbatim}
 chmod +x /usr/local/bin/pulledpork.pl
\end{verbatim}

È possibile verficare il corretto funzionamento eseguendo PulledPork:

\begin{verbatim}
/usr/local/bin/pulledpork.pl –V
\end{verbatim}

Il file di configurazione si trova invece in /etc/snort e si chiama pulledpork.conf. Bisogna effettuare alcune modifiche, quindi si legge il file con il comando
\texttt{vi /etc/snort/pulledpork.conf}

Viene specificato il path delle regole processate da snort.

\begin{verbatim}
rule_path=/usr/local/etc/snort/rules/snort.rules
\end{verbatim}

Pulled pork richiede anche il path delle local.rules per costruire i file sid-msg.map che contengono informazioni sui metadati (msg) di local.rules
\begin{verbatim}
local_rules=/usr/local/etc/snort/rules/local.rules
\end{verbatim}

I file sid-msg.map si trovano qui:
\begin{verbatim}
sid_msg=/usr/local/etc/snort/sid-msg.map
\end{verbatim}

viene specificato anche il path del file di configurazione
\begin{verbatim}
config_path=/usr/local/etc/snort/snort.conf
\end{verbatim}

Pulled pork aggiorna una blocklist di ip pubblici bloccati dalla comunità
\begin{verbatim}
block_list=/usr/local/etc/snort/rules/iplists/default.blocklist
\end{verbatim}

\section{Il demone snortd}

Per far si che Snort lavori in background viene attivato il demone snortd. I demoni sono dei programmi eseguiti in background, cioè senza il controllo diretto di un utente. Un sistema di difesa ha l'esigenza di catturare pacchetti e analizzare il traffico durante tutto l'arco di vita della macchina (esposta in rete). Per questo viene largamente sfruttata questa possibilità.

\begin{verbatim}
systemctl daemon-reload
systemctl start snortd
\end{verbatim}

Per avere la conferma che l'avvio del demone sia andata a buon fine:

\begin{verbatim}
systemctl status snortd
\end{verbatim}

\section{FWSnort}

FWsnort è uno script perl che traduce le regole di Snort in iptables. Alcune regole non hanno una traduzione diretta, tuttavia circa il 65 per cento delle rules possono essere tradotte con successo utilizzando l'iptables string match module (modulo che confronta l'inizio della stringa che compone la regola con la chiave di una mappa. Se viene trovata una corrispondenza restituisce l'oggetto corrispondete, un iptables). FWsnort analizza anche la policy iptables in esecuzione sulla macchina per determinare quali regole Snort sono applicabili.

\begin{verbatim}
cd /usr/local/src
wget http://www.cipherdyne.org/fwsnort/download/fwsnort-1.0.tar.bz2 
wget http://www.cipherdyne.org/fwsnort/download/fwsnort- 1.0.tar.bz2.md5
wget http://www.cipherdyne.org/fwsnort/download/fwsnort- 1.0.tar.bz2.asc
\end{verbatim}

\begin{verbatim}
tar xfj fwsnort-1.0.tar.bz2
cd /usr/local/src/fwsnort-1.0 
./install.pl
\end{verbatim}

Per far partire
\begin{verbatim}
fwsnort
\end{verbatim}

FWSnort se installato su un sistema operativo che offre il supporto per il string match module nel kernel, procederà per la traduzione delle regole in iptables, mostrando da cli quante regole sono state convertite correttamente.

Il file di configurazione si trova nella directory \texttt{/etc/fwsnort/fwsnort.conf}

FWSnort gestisce tutto il traffico che attraversa la macchina locale ed è diretto verso/proviene dalla lan. Nel file di configurazione vengono definite delle variabili che corrispondo a ip/porte e vengono utilizzate nelle rules.

\begin{verbatim}
HOME_NET                10.222.111.0/24;
EXTERNAL_NET            !$HOME_NET;

HTTP_SERVERS            $HOME_NET;
SMTP_SERVERS            $HOME_NET;
DNS_SERVERS             $HOME_NET;
SQL_SERVERS             $HOME_NET;
TELNET_SERVERS          $HOME_NET;

SSH_PORTS               [64022,65022];
HTTP_PORTS              80;
SHELLCODE_PORTS         !80;
ORACLE_PORTS            1521;
\end{verbatim}

Un esempio di community rules:

\begin{verbatim}
alert tcp $EXTERNAL_NET any -> $HTTP_SERVERS $HTTP_PORTS (msg:"SERVER-APACHE Apache Tomcat view source attempt"; flow:to_server,established; content:"%252ejsp"; http_uri; metadata:ruleset community, service http; reference:bugtraq,2527; reference:cve,2001-0590; classtype:web-application-attack; sid:1056; rev:16;)
\end{verbatim}