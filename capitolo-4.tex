\section{Progetto}

Dopo aver analizzato i requisiti presentati nel capitolo 3 si inizia a realizzare il progetto.

\subsection{Configurazione della rete}

Si procede come prima cosa con la suddivisione della rete per aree funzionali. La rete deve:

• gestire la sicurezza

• esporre i servizi su internet

Queste aree di lavoro vanno tenute separate per soddisfare due dei requisiti non funzionali indicati dall'azienda: la modularità e la scalabilità.
Si sceglie quindi di introdurre una seconda macchina virtuale chiamata vm victim che ospiti web server, database server e file server. Il nome 'victim' è giustificato dal fatto che ogni possibile attacco in rete sarà direzionato verso questa macchina virtuale, poiché espone dei servizi su internet. Si intende assegnare alla vm firewall ogni responsabilità relativa alla gestione della sicurezza dell'intera rete (firewall e altri strumenti di difesa...), mentre alla vm victim ogni responsabilità circa la gestione di dati e servizi.
Si intuisce che la victim non possa avere un ip nell'intervallo  10.222.111.0/24. Se così fosse sarebbe in grado di comunicare direttamente con il router telecom, invece l'obiettivo é che la victim si trovi \textbf{a valle} della vm firewall, in modo che questa possa filtrare il traffico.
Si decide di procedere in questo modo:

• Viene aggiunta una seconda scheda di rete ens224 alla vm firewall, con ip 50.50.50.1

• La vm victim appartiene alla lan 50.50.50.0/24, una lan di server, ha un'unica interfaccia ens160 (oltre a quella di loopback) che corrisponde all'indirizzo 50.50.50.3.  il router di default è 50.50.50.1.

Il flusso di traffico che arriva al router telecom viene rigirato sull'interfaccia esterna della vm firewall (ens160). Dopo essere stato filtrato/ analizzato la vm firewall svolge il ruolo di router per la lan 50.50.50.0/24, ossia smista i pacchetti tra i vari host seguendo le regole di forwarding. Questa configurazione è perfetta per soddisfare i requisiti di modularità e scalabilità e per permettere il corretto funzionamento della rete.
Se la vm victim riscontrasse qualche problema e fosse necessario un intervento, non c'è modo che questo possa interferire o compromettere la sicurezza dell'intera rete, poiché sulla vm victim non risiede alcuno script/ file/ servizio per la gestione della sicurezza. Le macchine virtuali sono separate per ruoli, e questo permette di non propagare modifiche e correzioni, bensì qualsiasi intervento ricadrà unicamente nell'aria di competenza.
Una rete realizzata in questo modo è anche scalabile. Se nel prossimo futuro l'azienda decidesse di espandersi e avesse bisogno di creare altre macchine virtuali, potrebbe popolare la lan dei server con ben altri 252 dispositivi! (256 indirizzi disponibili in /24 -1 prefisso -1 broadcast -1 indirizzo router - 1 indirizzo vm victim = 252).
Il cuore dell'intera rete è la lan dei server, su cui vengono installati dei servizi che sono esposti su internet. Per evitare perdita e compromissione dei dati, tentativi di connessione non autorizzati, injection di codice, e altra attività malevola bisogna lavorare sulla vm firewall, macchina virtuale il cui unico scopo è quello di proteggere la lan dei server. Configurare manualmente il firewall risolve solo parte dei problemi legati alla sicurezza. Si limita ad applicare delle regole, senza nessun tipo di consapevolezza sul traffico che sta attraversando la rete in quel momento. È possibile aspirare ad un tipo di intelligenza che sia superiore a quella del firewall, che permetta di effettuare un controllo sul contenuto dei singoli pacchetti, in modo da monitorare continuamente il flusso di traffico e rilevare e prevenire eventuali comportamenti anomali?

\begin{figure}[htb]
    \begin{center}
        \begin{tabular}{l}
            \includegraphics[width=15cm]{figure/net_final.pdf}
        \end{tabular}
    \end{center}
    \caption{Progetto di realizzazione della rete che soddisfi i requisiti imposti nel capitolo 3.}
\end{figure}

Vengono elencati brevemente i vari componenti della rete:

• vm firewall che ospita il firewall,

– IP 10.222.111.2, ens160, il router di default è 10.222.111.1

– IP 50.50.50.1, ens224

• vm victim nella lan dei server,

– IP 50.50.50.3, ens160, il router di default è 50.50.50.1

• Un altro componente della rete è uno switch virtuale

\vspace*{1cm}

Viene riportato brevemente uno scenario parallelo di insuccesso, per rafforzare ancora di piú la scelta di progettare la rete nel modo illustrato dalla figura.
Se la vm victim e la vm firewall fossero due host appartenenti alla stessa lan (e non rispettivamente un host e il router), la vm victim dovrebbe ospitare un firewall per filtrare il traffico proveniente dal router di default (il router telecom).